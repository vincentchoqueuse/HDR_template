%%Shorter bottom margin for the back cover
\newgeometry{inner=30mm,outer=20mm,vmargin=20mm}

%background image for resume (back)
\coverheader{./couverture/image-fond-MATHSTIC-dos.pdf}

% Switch font style to back cover style
\selectfontbackcover{ % Font style change is limited to this page using braces, just in case

\titleFR{Apports des techniques de traitement du signal paramétriques pour l’analyse des signaux électriques et les communications optiques cohérentes}

\keywordsFR{Traitement du Signal, Estimation et Détection, Réseaux Electriques Intelligents, Communications Optiques}

\abstractFR{
Ce mémoire résume mes activités de recherches menées ces 12 dernières années à l'IRDL puis au Lab-STICC.  Dans un premier temps, il présente une synthèse des travaux réalisés au sein de l'IRDL sur la période 2008-2018. Ces travaux couvrent essentiellement deux problématiques : le diagnostic des machines électriques et la surveillance des réseaux électriques (\emph{smart-grid}). Pour résoudre ces problématiques, l'originalité de nos travaux réside dans l'exploitation systématique de la structure signaux au moyen de techniques de traitement du signal paramétriques.

Dans un second temps, ce mémoire présente les activités de recherches initiées au laboratoire Lab-STICC depuis mon intégration en 2018. Ces activités concernent la conception d'algorithmes pour la compensation des imperfections dans les chaines de communications optiques cohérentes. Après avoir lister nos travaux en cours portant sur l'utilisation des approches paramétriques pour la compensation des imperfections linéaires au sens large (imperfections du laser, déséquilibre IQ et dispersion chromatique), ce mémoire insiste sur le potentiel des approches mixtes paramétriques / \emph{machine learning} pour compenser conjointement les imperfections linéaires et non-linéaires de la chaine de communication.
}



\titleEN{On the contribution of parametric signal processing algorithms for electrical signal monitoring and coherent optical communications}

\keywordsEN{Signal Processing, Estimation and Detection, Smart Grid, Optical Communication}

\abstractEN{
This manuscript presents my research activities carried out over the past 12 years at the laboratories IRDL (formerly LBMS) and Lab-STICC. First, it summarises my scientific contributions during the period 2008-2018 within the IRDL. These contributions essentially focus on two main topics :~the monitoring of electrical machines and the monitoring of electrical signals in smart-grids. To address these topics, the originality of our contributions lies in the use of parametric signal processing techniques.

Second, this manuscript describes the research activities conducted within the Lab-STICC since my integration in 2018. These activities focus on the  design of digital algorithms for imperfection compensation in coherent optical communication systems. After reviewing our current works dealing with the use of parametric approaches for the compensation of widely linear impairments (laser impermanent, IQ imbalance and chromatic dispersion), this thesis emphasizes the benefit of using mixed parametric approaches / machine learning to jointly compensate linear and non-linear impairments in optical communication systems.
}

